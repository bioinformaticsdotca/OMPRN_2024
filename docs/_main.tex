% Options for packages loaded elsewhere
\PassOptionsToPackage{unicode}{hyperref}
\PassOptionsToPackage{hyphens}{url}
%
\documentclass[
]{book}
\usepackage{amsmath,amssymb}
\usepackage{iftex}
\ifPDFTeX
  \usepackage[T1]{fontenc}
  \usepackage[utf8]{inputenc}
  \usepackage{textcomp} % provide euro and other symbols
\else % if luatex or xetex
  \usepackage{unicode-math} % this also loads fontspec
  \defaultfontfeatures{Scale=MatchLowercase}
  \defaultfontfeatures[\rmfamily]{Ligatures=TeX,Scale=1}
\fi
\usepackage{lmodern}
\ifPDFTeX\else
  % xetex/luatex font selection
\fi
% Use upquote if available, for straight quotes in verbatim environments
\IfFileExists{upquote.sty}{\usepackage{upquote}}{}
\IfFileExists{microtype.sty}{% use microtype if available
  \usepackage[]{microtype}
  \UseMicrotypeSet[protrusion]{basicmath} % disable protrusion for tt fonts
}{}
\makeatletter
\@ifundefined{KOMAClassName}{% if non-KOMA class
  \IfFileExists{parskip.sty}{%
    \usepackage{parskip}
  }{% else
    \setlength{\parindent}{0pt}
    \setlength{\parskip}{6pt plus 2pt minus 1pt}}
}{% if KOMA class
  \KOMAoptions{parskip=half}}
\makeatother
\usepackage{xcolor}
\usepackage{longtable,booktabs,array}
\usepackage{calc} % for calculating minipage widths
% Correct order of tables after \paragraph or \subparagraph
\usepackage{etoolbox}
\makeatletter
\patchcmd\longtable{\par}{\if@noskipsec\mbox{}\fi\par}{}{}
\makeatother
% Allow footnotes in longtable head/foot
\IfFileExists{footnotehyper.sty}{\usepackage{footnotehyper}}{\usepackage{footnote}}
\makesavenoteenv{longtable}
\usepackage{graphicx}
\makeatletter
\def\maxwidth{\ifdim\Gin@nat@width>\linewidth\linewidth\else\Gin@nat@width\fi}
\def\maxheight{\ifdim\Gin@nat@height>\textheight\textheight\else\Gin@nat@height\fi}
\makeatother
% Scale images if necessary, so that they will not overflow the page
% margins by default, and it is still possible to overwrite the defaults
% using explicit options in \includegraphics[width, height, ...]{}
\setkeys{Gin}{width=\maxwidth,height=\maxheight,keepaspectratio}
% Set default figure placement to htbp
\makeatletter
\def\fps@figure{htbp}
\makeatother
\setlength{\emergencystretch}{3em} % prevent overfull lines
\providecommand{\tightlist}{%
  \setlength{\itemsep}{0pt}\setlength{\parskip}{0pt}}
\setcounter{secnumdepth}{5}
\usepackage{booktabs}

\usepackage{color}
\usepackage{framed}
\setlength{\fboxsep}{.8em}

% These colours were manually entered, they shouldn't matter unless you want pdf output

\newenvironment{redbox}{
  \definecolor{shadecolor}{RGB}{243, 154, 157}
  \color{white}
  \begin{shaded}}
 {\end{shaded}}

\newenvironment{bluebox}{
  \definecolor{shadecolor}{RGB}{172, 210, 237}
  \color{white}
  \begin{shaded}}
 {\end{shaded}}

\newenvironment{greenbox}{
  \definecolor{shadecolor}{RGB}{141, 181, 128}
  \color{white}
  \begin{shaded}}
 {\end{shaded}}
\ifLuaTeX
  \usepackage{selnolig}  % disable illegal ligatures
\fi
\usepackage[]{natbib}
\bibliographystyle{plainnat}
\usepackage{bookmark}
\IfFileExists{xurl.sty}{\usepackage{xurl}}{} % add URL line breaks if available
\urlstyle{same}
\hypersetup{
  pdftitle={Bridging Pathology and Genomics: A Practical Workshop on NGS for Pathologists and Pathology Researchers 2024},
  pdfauthor={Faculty: INSTRUCTOR AND TA NAMES},
  hidelinks,
  pdfcreator={LaTeX via pandoc}}

\title{Bridging Pathology and Genomics: A Practical Workshop on NGS for Pathologists and Pathology Researchers 2024}
\author{Faculty: INSTRUCTOR AND TA NAMES}
\date{October 21, 2024 - October 22, 2024}

\begin{document}
\maketitle

{
\setcounter{tocdepth}{1}
\tableofcontents
}
\part{Introduction}\label{part-introduction}

\chapter{Workshop Info}\label{workshop-info}

Welcome to the 2024 Bridging Pathology and Genomics: A Practical Workshop on NGS for Pathologists and Pathology Researchers Canadian Bioinformatics Workshop webpage!

\section{Class Photo}\label{class-photo}

\section{Schedule}\label{schedule}

\begin{longtable}[]{@{}
  >{\centering\arraybackslash}p{(\columnwidth - 6\tabcolsep) * \real{0.1250}}
  >{\centering\arraybackslash}p{(\columnwidth - 6\tabcolsep) * \real{0.3500}}
  >{\centering\arraybackslash}p{(\columnwidth - 6\tabcolsep) * \real{0.1250}}
  >{\centering\arraybackslash}p{(\columnwidth - 6\tabcolsep) * \real{0.4000}}@{}}
\toprule\noalign{}
\begin{minipage}[b]{\linewidth}\centering
\textbf{Time (Eastern)}
\end{minipage} & \begin{minipage}[b]{\linewidth}\centering
\textbf{October 21, 2024}
\end{minipage} & \begin{minipage}[b]{\linewidth}\centering
\textbf{Time (Eastern)}
\end{minipage} & \begin{minipage}[b]{\linewidth}\centering
\textbf{October 22, 2024}
\end{minipage} \\
\midrule\noalign{}
\endhead
\bottomrule\noalign{}
\endlastfoot
08:30 - 09:00 & Arrivals \& Check-in & 08:30 - 09:00 & Arrivals \\
9:00 - 9:30 & Welcome (Nia Hughes) & 9:00 - 9:30 & Recap of Day 1 and Introduction to Day 2 (Larry) \\
9:30 - 10:30 & Sequencing Platforms and Technologies (Bernard) & 9:30 - 11:00 & Sequence Analysis (Larry) \\
10:30 - 11:00 & Break & 11:00 - 11:30 & Break \\
11:00 - 12:00 & Laboratory Operations and Sample Preparation (Bernard) & 11:30 - 13:00 & Hands-on Exercise: Sequence Quality and Variant Review (Larry) \\
12:00 - 13:00 & Lunch Break & 13:00 - 14:00 & Lunch Break \\
13:00 - 14:00 & Hands-on Exercise/Discussion (Bernard) & 14:00 - 15:00 & Clinical Reporting pt.~1 (Iain) \\
14:00 - 15:30 & Lab Tours (Bernard) & 15:00 - 15:30 & Break \\
15:30 - 16:00 & Break & 15:30 - 16:00 & Clinical Reporting pt.~2 (Iain) \\
16:00 - 17:00 & Data Generated by Different Platforms (Larry) & 16:00 - 17:00 & Hands-on Exercises (Iain) \\
17:00 - 17:30 & Q\&A and Wrap-Up (Day 1) & 17:00 - 17:30 & Q\&A and Wrap-Up (Day 2) \\
17:30 & Finished & 17:30 & Finished \\
\end{longtable}

\section{Pre-work}\label{pre-work}

\href{https://docs.google.com/forms/d/e/1FAIpQLScewZhdlVSzXpY77kyWicauKDaEQy37RW4ZPZ9KNYWJyB03Mg/viewform}{You can find your pre-work here.}

\chapter{Data and Compute Setup}\label{data-and-compute-setup}

\subsubsection{Course data downloads}\label{course-data-downloads}

Coming soon!

\subsubsection{Compute setup}\label{compute-setup}

Coming soon!

\part{Modules}\label{part-modules}

\chapter{Module 1}\label{module-1}

\section{Lecture}\label{lecture}

\section{Lab}\label{lab}

\chapter{Module 2}\label{module-2}

\section{Lecture}\label{lecture-1}

\section{Lab}\label{lab-1}

\chapter{Module 3}\label{module-3}

\section{Lecture}\label{lecture-2}

\section{Lab}\label{lab-2}

\chapter{Module 4}\label{module-4}

\section{Lecture}\label{lecture-3}

\section{Lab}\label{lab-3}

  \bibliography{book.bib,packages.bib}

\end{document}
